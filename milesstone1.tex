\documentclass[conference]{IEEEtran}
\IEEEoverridecommandlockouts
% The preceding line is only needed to identify funding in the first footnote. If that is unneeded, please comment it out.
\usepackage{apacite}
\usepackage{amsmath,amssymb,amsfonts}
\usepackage{algorithmic}
\usepackage{graphicx}
\usepackage{textcomp}
\usepackage{xcolor}
\def\BibTeX{{\rm B\kern-.05em{\sc i\kern-.025em b}\kern-.08em
    T\kern-.1667em\lower.7ex\hbox{E}\kern-.125emX}}
\begin{document}

\title{FlexRay Communication Protocol\\
}

\author{\IEEEauthorblockN{1\textsuperscript{st} Oguz Kaan Tuna}
\IEEEauthorblockA{\textit{Hochschule Hamm-Lippstadt} \\
\textit{Electronic Engineering}\\
4th Semester \\
oguz-kaan.tuna@stud.hshl.de}
\and
\IEEEauthorblockN{2\textsuperscript{nd} Kristina Bala}
\IEEEauthorblockA{\textit{Hochschule Hamm-Lippstadt} \\
\textit{Electronic Engineering}\\
4th Semester \\
kristina.bala@stud.hshl.de}

}

\maketitle

\begin{abstract}

\end{abstract}

\begin{IEEEkeywords}

\end{IEEEkeywords}

\section{First References}
[1]\cite{enosh2014efficient}.

[2]\cite{kopetz2001comparison}.

[3]\cite{makowitz2006flexray}

[4]\cite{shaw2008introduction}

[5]\cite{rausch2007flexray}

[6]\cite{xu2008implementation}

\section{Introduction}


\section{FlexRay}

\section{History of FlexRay}

\section{Charachteristics of FlexRay}


\subsection{Applications}


\subsection{FlexRay Advantages}

\subsection{Network Topology}

\section{Communication Architecture}

\subsection{Communication Controller}

\section{Communucation Channel}


\subsection{Static Segment}

\subsection{Dynamic Segment}

\section{Differences Between Protocols}
\section*{Acknowledgment}

\bibliographystyle{apacite}
\bibliography{references}
\end{document}
